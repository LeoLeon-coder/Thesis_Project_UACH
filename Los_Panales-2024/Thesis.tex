\documentclass[12pt]{report}
\usepackage[spanish]{babel}
\usepackage[a4paper, total={7in, 9in}]{geometry}
\usepackage{setspace} %Line spacing
\onehalfspacing

\usepackage{pdfpages}

\usepackage{fancyhdr}
\setlength{\headheight}{15.71667pt} % Increased as recommended by fancyhdr
\addtolength{\topmargin}{-1.21667pt} % Optional: adjust topmargin as suggested
\usepackage{lastpage}

% For the chapters not to modify the page style
\usepackage{titlesec}
\titleformat{\chapter}{\bfseries}{\huge\arabic{chapter}.}{10pt}{\huge}
\titleclass{\chapter}{straight}

\usepackage{siunitx, multirow, booktabs}
\usepackage{blindtext,alltt}
\usepackage{csquotes} % Recommended for biblatex with babel/polyglossia
\usepackage[backend=biber, style=numeric, sorting=none, locallabelwidth]{biblatex}
\addbibresource{ThesisReferences.bib}

\usepackage{listings}

\usepackage{wrapfig}
\usepackage{array}
\usepackage{url}
\usepackage[nottoc]{tocbibind} %Add references to index.
\usepackage{amsmath, amssymb}
\usepackage{upgreek, dsfont, mathrsfs}
\usepackage{stmaryrd}

\usepackage{multicol}

% Estilo compacto para secciones en columnas
\titlespacing{\section}{0pt}{1ex plus 0.6ex minus 0.2ex}{0.8ex}
\titlespacing{\subsection}{0pt}{0.8ex plus 0.4ex minus 0.2ex}{0.5ex}
\titlespacing{\subsubsection}{0pt}{0.6ex plus 0.3ex minus 0.1ex}{0.4ex}

% Configuración de columnas tipo revista
\setlength{\columnsep}{25pt}           % separación entre columnas (default ~35pt)
\setlength{\columnseprule}{0pt}      % línea entre columnas (0pt = sin línea)

\usepackage{parskip} % Stop auto-indenting (to mimic markdown behaviour)
\usepackage[breakable]{tcolorbox}
% Compactar el interlineado en columnas (opcional)
\newenvironment{tightmulticols}{%
  \begin{multicols}{2}
  \setlength{\parskip}{0pt}
  \setlength{\parindent}{0em}
  \linespread{1}\selectfont
}{%
  \end{multicols}
}

% Basic figure setup, for now with no caption control since it's done
% automatically by Pandoc (which extracts ![](path) syntax from Markdown).
\usepackage{graphicx}
\usepackage{caption}

%Footnoe symbol instead of numbers
%1   asterisk        *   2   dagger      †   3   double dagger       ‡
%4   section symbol  §   5   paragraph   ¶   6   parallel lines      ‖
%7   two asterisks   **  8   two daggers ††  9   two double daggers  ‡‡
\usepackage[symbol]{footmisc}
\renewcommand{\thefootnote}{\fnsymbol{footnote}}

\usepackage{float}
\floatplacement{figure}{H} % forces figures to be placed at the correct location

\usepackage{xcolor} % Allow colors to be defined
\usepackage{enumerate} % Needed for markdown enumerations to work
\usepackage{textcomp} % defines textquotesingle
% Hack from http://tex.stackexchange.com/a/47451/13684:
\AtBeginDocument{%
	\def\PYZsq{\textquotesingle}% Upright quotes in Pygmentized code
}
\usepackage{upquote} % Upright quotes for verbatim code
\usepackage{eurosym} % defines \euro

\usepackage{iftex}
\ifPDFTeX
\IfFileExists{alphabeta.sty}{
	\usepackage{alphabeta}
}{
	\usepackage[mathletters]{ucs}
	\usepackage[utf8x]{inputenc}
}
\else
\usepackage{fontspec}
\usepackage{unicode-math}
\fi


% The hyperref package gives us a pdf with properly built
% internal navigation ('pdf bookmarks' for the table of contents,
% internal cross-reference links, web links for URLs, etc.)
\usepackage{hyperref}
% The default LaTeX title has an obnoxious amount of whitespace. By default,
% titling removes some of it. It also provides customization options.
\usepackage{titling}
\usepackage{longtable} % longtable support required by pandoc >1.10
\usepackage{calc}      % table minipage width calculation for pandoc >= 2.11.1
\usepackage[inline]{enumitem} % IRkernel/repr support (it uses the enumerate* environment)
\usepackage[normalem]{ulem} % ulem is needed to support strikethroughs (\sout)
% normalem makes italics be italics, not underlines
\usepackage{mathrsfs}

% More styles for bullets
\usepackage{pifont}

% Colors for the hyperref package
\definecolor{urlcolor}{rgb}{0,.145,.698}
\definecolor{linkcolor}{rgb}{.71,0.21,0.01}
\definecolor{citecolor}{rgb}{.12,.54,.11}

% ANSI colors
\definecolor{ansi-black}{HTML}{3E424D}
\definecolor{ansi-black-intense}{HTML}{282C36}
\definecolor{ansi-red}{HTML}{E75C58}
\definecolor{ansi-red-intense}{HTML}{B22B31}
\definecolor{ansi-green}{HTML}{00A250}
\definecolor{ansi-green-intense}{HTML}{007427}
\definecolor{ansi-yellow}{HTML}{DDB62B}
\definecolor{ansi-yellow-intense}{HTML}{B27D12}
\definecolor{ansi-blue}{HTML}{208FFB}
\definecolor{ansi-blue-intense}{HTML}{0065CA}
\definecolor{ansi-magenta}{HTML}{D160C4}
\definecolor{ansi-magenta-intense}{HTML}{A03196}
\definecolor{ansi-cyan}{HTML}{60C6C8}
\definecolor{ansi-cyan-intense}{HTML}{258F8F}
\definecolor{ansi-white}{HTML}{C5C1B4}
\definecolor{ansi-white-intense}{HTML}{A1A6B2}
\definecolor{ansi-default-inverse-fg}{HTML}{FFFFFF}
\definecolor{ansi-default-inverse-bg}{HTML}{000000}

% common color for the border for error outputs.
\definecolor{outerrorbackground}{HTML}{FFDFDF}

% Prevent overflowing lines due to hard-to-break entities
\sloppy 
% Setup hyperref package
\hypersetup{
	breaklinks=true,  % so long urls are correctly broken across lines
	colorlinks=true,
	urlcolor=urlcolor,
	linkcolor=linkcolor,
	citecolor=citecolor,
}

\definecolor{codegreen}{rgb}{0,0.6,0}
\definecolor{codegray}{rgb}{0.5,0.5,0.5}
\definecolor{codepurple}{rgb}{0.58,0,0.82}
\definecolor{backcolour}{rgb}{0.95,0.95,0.92}

\lstdefinestyle{mystyle}{
	backgroundcolor=\color{backcolour},   
	commentstyle=\color{codegreen},
	keywordstyle=\color{magenta},
	numberstyle=\tiny\color{codegray},
	stringstyle=\color{codepurple},
	basicstyle=\ttfamily\footnotesize,
	breakatwhitespace=false,         
	breaklines=true,                 
	captionpos=b,                    
	keepspaces=true,                 
	numbers=left,                    
	numbersep=5pt,                  
	showspaces=false,                
	showstringspaces=false,
	showtabs=false,                  
	tabsize=2
}

\lstset{style=mystyle}

\pagestyle{fancy}
\fancyhf{}
\rhead{}
\lhead{APRENDIZAJE PROFUNDO $|$ \textcolor{orange}{DETECCIÓN DE LA ABEJA REINA EN LA COLMENA}}
\lfoot{Universidad Autónoma de Chihuahua}
\rfoot{Page \thepage \hspace{1pt} of \pageref{LastPage}}

\renewcommand{\headrulewidth}{0.5pt} %ancho de la recta del encabezado superior
\begin{document}
	
	\begin{titlepage}
		\begin{center}
			\begin{tabular}{c}
				\includegraphics[scale=0.2]{BN_uach.png}\\[3.5ex]
				\textbf{\LARGE Universidad Autónoma de Chihuahua}\\[3.5ex]
				\textbf{\Large Facultad de Ingeniería}\\[3.5ex]
				\hline\\[3ex]
				\begin{minipage}{17cm}
					\centering
					\begin{doublespace}
						\textbf{\LARGE Detección de la Abeja Reina en una Colmena por Medio del Aprendizaje Profundo}\\[3.5ex]
					\end{doublespace}
				\end{minipage}\\[3.5ex]
				\hline
			\end{tabular}\vfill
			{\large Análisis de Frecuencias Emitidas por el Panal}\\\vfill
			{\large \textbf{Student:} Leonardo Rafael León Mora.}\\\vfill
			{\large \textbf{Thesis director:} Dr. Daniel Espinobarro Velázquez.}\\\vfill
			{\large \textbf{Advisors:}}\\[3.5ex]
			\begin{itemize}
				\item[\ding{226}] {\large M.C. Carlos Hugo Larrinúa Pacheco.}\\[3.5ex]
				\item[\ding{226}] {\large M.I. Joseph Isaac Ramírez Hernández.}\\[3.5ex]
			\end{itemize}
			\vfill
			{\large \textbf{Chihuahua, Chih.,} \today.}\\[3.5ex]
		\end{center}
	\end{titlepage}

%-------------------------------------------------------------------------------------
% ÍNDICE
%-------------------------------------------------------------------------------------
\tableofcontents
\thispagestyle{empty} % No header/footer on the table of contents page


%-------------------------------------------------------------------------------------
% RESUMEN/ABSTRACT
%-------------------------------------------------------------------------------------
\pagebreak
\chapter*{RESUMEN}
\addcontentsline{toc}{chapter}{RESUMEN}

%-------------------------------------------------------------------------------------
% 1. INTRODUCCIÓN
%-------------------------------------------------------------------------------------
\pagebreak
\chapter{INTRODUCCIÓN}
\vspace{-3em}

\begin{tightmulticols}
Las abejas melíferas han sido ampliamente estudiadas por su destacada contribución al medio ambiente, proporcionando recurosos que son vitales para animales, plantas y seres humanos. Su papel en la polinización es crucial para la biodiversidad y la producción agrícola\cite{leon2023calidad}. Además, la miel, el polen, la jalea real y la cera de abejas son productos altamente valorados por sus propiedades nutricionales y medicinales\cite{winston1987temperate}.\\
\hspace{2em}
\par Durante las temporadas de mayor producción de miel, como en primavera y verano, las abejas melíferas pueden recorrer distancias de hasta 6 km\cite{Beekman2000}, viajando aproximadamente 3 km de ida y vuelta entre la colmena y las fuentes de alimento, como flores, para recolectar polen, néctar o agua. Pero, ¿por qué surge este comportamiento? ¿Cuál es la motivación de las abejas para regresar a la colmena y abastecerla?
\par La comunicación dentro de la colmena es esencial para la supervivencia de la colonia. Las abejas utilizan una variedad de señales, incluyendo feromonas y sonidos, para coordinar sus actividades y mantener el orden social\cite{winston1987temperate}. La presencia de una abeja reina es fundamental para la cohesión del grupo, ya que su ausencia puede conducir a un comportamiento caótico y al eventual colapso de la colonia\cite{ruvinga2021use}. Otra forma de comunicación es el \textit{waggle dance}, mediante el cual las abejas pueden informar sobre la ubicación de fuentes de alimento, tomando como referencia los puntos cardinales y trazando un vector en la dirección requerida\cite{wang2025encoding}, lo que permite notificar a las demás abejas recolectoras.\\

\par Según Ruvinga\cite{ruvinga2021use}, ``\textit{La colonia está formada por la reina, miles de abejas obreras estériles, unos pocos cientos de zánganos, huevos, larvas y pupas}''. Las abejas trabajan a diario, realizando tareas que se dividen no solo entre machos y hembras, sino también entre abejas jóvenes y adultas\cite{johnson2010division}. Un panal puede albergar entre 30,000 y 50,000 abejas, aunque este número puede reducirse hasta 5,000 durante el invierno. Este comportamiento de abastecimiento constante está estrechamente vinculado a la estructura social de la colmena y, en particular, al papel de la abeja reina, que es la única capaz de procrear\cite{ruvinga2021use}.

%-------------------------------------------------------------------------------------
% 1.1. Planteamiento del problema
%-------------------------------------------------------------------------------------

\section{Planteamiento del Problema}

\par La reina puede poner en promedio hasta 2,000 huevos al día. Las obreras se encargan de construir, defender y abastecer la colmena con recursos esenciales como agua, polen y néctar\cite{winston1987temperate}. Los zánganos cumplen funciones reproductivas y también se cree que ayudan a regular la temperatura del panal, mientras que las abejas más jóvenes se limitan a tareas dentro de la colmena\cite{johnson2010division}. La pérdida de la abeja reina puede comprometer seriamente la supervivencia de la colonia.

\par Las colmenas están expuestas a múltiples amenazas: insectos invasores, pesticidas, condiciones climáticas y hasta errores humanos. En sistemas apícolas tradicionales donde se utilizan cajas de madera, cada inspección física implica abrir la colmena y romper el sello de propóleo que las abejas han construido para mantener un ambiente limpio y controlado. Esta intervención frecuente no solo representa una pérdida energética para la colonia (que debe reconstruir el sello), sino que además incrementa el riesgo de aplastar accidentalmente a la reina durante la manipulación interna del panal. Esta situación puede conducir al colapso de toda la colonia.

%-------------------------------------------------------------------------------------
% 1.2. Justificación
%-------------------------------------------------------------------------------------

\section{Justificación}

\par Contar con herramientas que permitan identificar la presencia o ausencia de la abeja reina de manera no invasiva resulta fundamental para garantizar la continuidad y productividad de la colmena. Esta investigación propone un enfoque basado en el análisis acústico del panal, aprovechando técnicas de aprendizaje automático para detectar el estado de la reina. Dado que estudios recientes han demostrado que el sonido dentro de la colmena varía según la presencia o ausencia de la reina\cite{ruvinga2021use, maarefremote, hunter2013signal, ruvinga2023identifying}, se busca validar un sistema que, a partir de grabaciones de audio, identifique estas variaciones de forma automática y eficiente.

%-------------------------------------------------------------------------------------
% 1.3. Objetivos
%-------------------------------------------------------------------------------------

\section{Objetivos}

\subsection{Objetivo General}

\par Desarrollar un sistema de detección automática con respecto a la presencia de la abeja reina en una colmena a partir de la recolección de datos, el análisis de audio y la implementación de técnicas de aprendizaje automático. Se pretende obtener un modelo capaz de clasificar el estado de la reina en función de las características acústicas extraídas de grabaciones de audio. Este enfoque no invasivo busca minimizar la perturbación de la colonia durante las inspecciones recurrentes que suele hacer un apicultor, preservando así el bienestar de las abejas y la integridad del panal.

\subsection{Objetivos Específicos}

\begin{itemize}
    \item[\ding{43}] Realizar una revisión bibliográfica sobre la estructura de las colonias de abejas melíferas y los efectos de la ausencia de la reina.
    \item[\ding{43}] Procesar y organizar un conjunto de datos de audio de colmenas etiquetado con el estado de la reina.
    \item[\ding{43}] Extraer características acústicas relevantes, como los coeficientes cepstrales en la escala Mel (MFCC), a partir de las grabaciones.
    \item[\ding{43}] Entrenar y evaluar modelos de aprendizaje automático para clasificar el estado de la reina basado en características extraídas.
    \item[\ding{43}] Validar la efectividad del sistema en términos de precisión, sensibilidad y especificidad.
\end{itemize}

\section{Hipótesis de Trabajo}

\par Dado que el comportamiento sonoro de una colmena varía según el estado de su reina, se espera que un sistema basado en el análisis de características acústicas sea capaz de detectar, con un grado aceptable de precisión, si la reina está presente, ausente, rechazada o recién aceptada en la colonia.

\end{tightmulticols}

%-------------------------------------------------------------------------------------
% 2. MARCO TEÓRICO
%-------------------------------------------------------------------------------------

\pagebreak
\chapter{MARCO TEÓRICO}
\vspace{-3em}

\begin{tightmulticols}

%-------------------------------------------------------------------------------------
% 2.1. Definición de conceptos clave
%-------------------------------------------------------------------------------------

\section{Conceptos Clave}

\begin{itemize}
    \item[\ding{43}] \textbf{Colmena}: estructura donde viven las abejas melíferas, compuesta por una reina, obreras y zánganos.
	\item[\ding{43}] \textbf{Reina}: abeja hembra fértil encargada de la reproducción en la colmena.
	\item[\ding{43}] \textbf{Razón de muestreo (Sample Rate)}: número de muestras de audio tomadas por segundo (cuyos valores para el monitoreo de abejas comunmente rondan entre los 16000 a 22050 Hz).
	\item[\ding{43}] \textbf{MFCC (Mel-Frequency Cepstral Coefficients)}: características acústicas utilizadas para representar el contenido espectral de una señal de audio, especialmente útiles en el reconocimiento de patrones sonoros.
\end{itemize}

%-------------------------------------------------------------------------------------
% Revisión de literatura
%-------------------------------------------------------------------------------------

%-------------------------------------------------------------------------------------
% Fundamentos teóricos
%-------------------------------------------------------------------------------------


\end{tightmulticols}

%-------------------------------------------------------------------------------------
% 3. METODOLOGÍA
%-------------------------------------------------------------------------------------

\pagebreak
\chapter{METODOLOGÍA}
\vspace{-3em}

\begin{tightmulticols}

%-------------------------------------------------------------------------------------
% 3.1. Procedimiento
%-------------------------------------------------------------------------------------
\section{Procedimiento}

\par En este estudio, se trabajó con audio monofónico (1 canal), formato PCM de 16 bits y una razón de muestreo de 22050 Hz, lo cual se determinó por medio del programa \ref{lst:waveform}. Posteriormente, se crearon diversos archivos *.ipynb a partir de los cuales se desarrollaba el código que corresponde a cada paso del programa final. Los resultados fueron el determinante para realizar modificaciones en las líneas de código, con el objetivo de aumentar la precisión de los resultados en cada corrida. Es decir, si los resultados no eran los esperados, con una presición de almenos el 90\%; se alteraban cualesquiera de los archivos de código en función de mejorar los mismos.

\par Inicialmente se llevó a cabo una prueba de extracción de características acústicas, específicamente los coeficientes cepstrales en la escala Mel (MFCC), con apoyo de la librería \texttt{librosa} (\ref{lst:test_mfcc}). De modo que si llegase a existir un error en la conversión del archivo de audio a un \texttt{*.csv} o \texttt{*.npy}, se pudiese detectar y corregir a tiempo.

\par En este programa de prueba se utilizó un archivo de audio (\url{2022-06-08--14-52-28_1__segment3.wav}), a partir del cual se obtuvo una matriz tiempo-frecuencia:
\begin{align*}
[\text{Time Frames}] \times [\text{13 MFCC Coefficients}] \; ,
\end{align*}
correspondiente a 13 coeficientes calculados por cada 25 milisegundos de audio, con un solapamiento del 50\% entre ventanas.

\subsection{1er Programa}

\par Estó porporcionó archivos que fueron exportados en formatos \texttt{.csv} y \texttt{.npy} para su posterior análisis.

%-------------------------------------------------------------------------------------
% 3.2. Instrumentos y técnicas
%-------------------------------------------------------------------------------------
\section{Instrumentos}

\par Para el proceso de recolección de datos, se empleó un dispositivo IoT diseñado a medida, que incluía un módulo Wi-Fi \texttt{ESP32}, un micrófono digital \texttt{INMP441}, y un sensor de temperatura y humedad \texttt{BME280}. Estos sensores se instalaron dentro de la cubierta telescópica de la colmena —por encima de los cuadros pero por debajo de la cubierta exterior—, y se protegieron con una malla metálica para evitar la interferencia de las abejas con los componentes electrónicos, especialmente el micrófono.

%-------------------------------------------------------------------------------------
% 3.3. Población y muestra
%-------------------------------------------------------------------------------------
\section{Base de Datos}

\par El conjunto de datos utilizado en esta investigación fue grabado y publicado por Anna Yang\cite{AnnaYang-Dataset}. Contiene datos de audio recolectados de colmenas de abejas melíferas europeas ubicadas en California. Las grabaciones fueron segmentadas en clips de 60 segundos, resultando en un total de 7,100 muestras recolectadas durante un período de cinco semanas, desde el 8 de junio de 2022 (14:52:28) hasta el 15 de julio de 2022 (15:28:21). Cada día contiene aproximadamente entre 23 y 24 muestras.

\par El conjunto de datos incluye múltiples variables de metadatos que describen el entorno de la colmena y la condición de la colonia. De particular relevancia para este trabajo se encuentran las siguientes etiquetas:

\begin{itemize}
	\item[\ding{43}] \texttt{queen acceptance}: indica si la reina es aceptada por la colonia. Valores: 0 (sin reina), 1 (presente pero no aceptada), 2 (aceptada).
	\item[\ding{43}] \texttt{queen presence}: indica si la reina está físicamente presente en la colmena.
	\item[\ding{43}] \texttt{queen status}: una etiqueta combinada derivada de las dos anteriores, con los siguientes valores: 0 (original/con reina funcional), 1 (no presente), 2 (presente y rechazada), 3 (presente y recién aceptada). Esta es la etiqueta principal utilizada para la clasificación.
	\item[\ding{43}] \texttt{time}: hora del día, escalada entre 0 y 1 para representar un ciclo de 24 horas.
\end{itemize}

\par El proceso de etiquetado se basó en inspecciones manuales diarias de las colmenas, durante las cuales se observó el comportamiento de las abejas hacia la reina (por ejemplo, formación de enjambre o conducta tranquila). En aquellos casos en los que el estado de la reina cambió entre inspecciones, generando ambigüedad sobre el momento exacto de aceptación, las muestras correspondientes fueron descartadas para mantener la integridad del etiquetado.

\end{tightmulticols}

%-------------------------------------------------------------------------------------
% 4. RESULTADOS
%-------------------------------------------------------------------------------------

\pagebreak
\chapter{RESULTADOS}
\vspace{-3em}

%-------------------------------------------------------------------------------------
% 4.1. Análisis de datos
%-------------------------------------------------------------------------------------

%-------------------------------------------------------------------------------------
% 4.2. Presentación de resultados
%-------------------------------------------------------------------------------------

%-------------------------------------------------------------------------------------
% 5. DISCUSIÓN
%-------------------------------------------------------------------------------------

\pagebreak
\chapter{DISCUSIÓN}
\vspace{-3em}

%-------------------------------------------------------------------------------------
% 5.1. Interpretación de resultados
%-------------------------------------------------------------------------------------

%-------------------------------------------------------------------------------------
% 5.2. Comparación con la literatura
%-------------------------------------------------------------------------------------

%-------------------------------------------------------------------------------------
% 6. CONCLUSIONES Y RECOMENDACIONES
%-------------------------------------------------------------------------------------
\pagebreak
\chapter{CONCLUSIONES Y RECOMENDACIONES}

%-------------------------------------------------------------------------------------
% 6.1. Conclusiones
%-------------------------------------------------------------------------------------

\section{Conclusiones}

\par La abjea reina es un elemento crucial para la supervivencia de la colonia, y su presencia o ausencia puede influir significativamente en el comportamiento de las abejas, siendo una posibilidad el extingir la existensia de la colmena.

%-------------------------------------------------------------------------------------
% 6.2. Recomendaciones
%-------------------------------------------------------------------------------------

%-------------------------------------------------------------------------------------
% REFERENCIAS BIBLIOGRÁFICAS
%-------------------------------------------------------------------------------------

\pagebreak
\addcontentsline{toc}{chapter}{BIBLIOGRAFÍA}
\printbibliography
\thispagestyle{empty}

%-------------------------------------------------------------------------------------
% ANEXOS (opcional)
%-------------------------------------------------------------------------------------

\pagebreak
\addcontentsline{toc}{chapter}{APÉNDICES}
\begin{center}
\large\textbf{APÉNDICE A}
\end{center}

\begin{lstlisting}[language=Python, caption={Visualización de la forma de onda de un archivo de audio.}, label={lst:waveform}]
import numpy as np
import matplotlib.pyplot as plt
from scipy.io import wavfile

# cargar archivo de audio
sr, audio = wavfile.read('2022-06-08--14-52-28_1__segment3.wav')

# normalizar si es int16
if audio.dtype == np.int16:
    audio = audio.astype(np.float32) / np.iinfo(np.int16).max

# crear eje de tiempo
duration_sec = len(audio) / sr
time_axis = np.linspace(0, duration_sec, num=len(audio))

# graficar la onda
plt.figure(figsize=(12, 4))
plt.plot(time_axis, audio)
plt.title(f'Onda (Sample Rate: {sr} Hz)')
plt.xlabel('Tiempo (segundos)')
plt.ylabel('Amplitud')
plt.grid(True)
plt.tight_layout()
plt.show()
\end{lstlisting}

\pagebreak
\begin{center}
\large\textbf{APÉNDICE A (continuación)}
\end{center}

\begin{lstlisting}[language=Python, caption={Prueba para extracción de características.}, label={lst:test_mfcc}]
import librosa
import numpy as np
import pandas as pd
import os

# CONFIG
audio_path = '<<archivo>>'
sample_rate = 22050
n_mfcc = 13

# CARGAR ARCHIVO
y, sr = librosa.load(audio_path, sr=sample_rate)

# EXTRAER MFCCs
mfccs = librosa.feature.mfcc(y=y, sr=sr, n_mfcc=n_mfcc)

# Transponer de modo que la forma resultante es: [frames x coefficients]
mfccs = mfccs.T

# Guardar en CSV
base_name = os.path.splitext(os.path.basename(audio_path))[0]
csv_output_path = f'{base_name}_mfcc.csv'
np_output_path = f'{base_name}_mfcc.npy'

pd.DataFrame(mfccs).to_csv(csv_output_path, index=False)

# O, guardar como un archivo binario de numpy
np.save(np_output_path, mfccs)

print(f'MFCCs en:\n- CSV: {csv_output_path}\n- NPY: {np_output_path}')
print(f'MFCCs forma: {mfccs.shape} to [frames x {n_mfcc}]')
\end{lstlisting}

\pagebreak
\begin{center}
\large\textbf{APÉNDICE A (continuación)}
\end{center}

\begin{lstlisting}[language=Python, caption={Extracción de características.}, label={lst:FE_wavtonpy}]
import os
import librosa
import numpy as np
import pandas as pd

# CONFIG
audio_folder = '<<arch. *.wav>>'
output_folder = '<<dir. guardar archs. npy>>'
csv_path = '<<arch. *.csv>>'
sample_rate = 22050
n_mfcc = 13

# Crear una carpeta de salida en caso de que no exista
os.makedirs(output_folder, exist_ok=True)

# Cargar info de etiquetas
labels_df = pd.read_csv(csv_path)
labels_df['file name'] = labels_df['file name'].str.strip()
labels_df.set_index('file name', inplace=True)

# Guardar el enlace con las etiquetas
metadata = []

# Procesando cada archivo de audio
for filename in os.listdir(audio_folder):
    if filename.endswith('.wav'):
        try:
            file_path = os.path.join(audio_folder, filename)
            y, sr = librosa.load(file_path, sr=sample_rate)
            mfcc = librosa.feature.mfcc(y=y, sr=sr, n_mfcc=n_mfcc)
            mfcc = mfcc.T  # Shape: [frames, 13]

            # Guardar matriz MFCC como .npy
            npy_name = filename.replace('.wav', '_mfcc.npy')
            np.save(os.path.join(output_folder, npy_name), mfcc)

			# Vincular con la etiqueta usando el nombre del archivo sin el segmento
            raw_key = filename.split('__')[0] + '.raw'
            label = labels_df.loc[raw_key]['queen status'] if raw_key in labels_df.index else None

            metadata.append({'file_name': filename, 'mfcc_file': npy_name, 'queen_status': label})

        except Exception as e:
            print(f'Error processing {filename}: {e}')

# Guardar archivo de metadatos
pd.DataFrame(metadata).to_csv('mfcc_metadata.csv', index=False)

print(f'{len(metadata)} archivos han sido procesados. MFCCs guardados en '{output_folder}'')
\end{lstlisting}

\end{document}