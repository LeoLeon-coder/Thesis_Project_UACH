\documentclass[12pt]{report}
\usepackage[spanish]{babel}
\usepackage[a4paper, total={7in, 9in}]{geometry}
\usepackage{setspace} %Line spacing
\onehalfspacing

\usepackage{pdfpages}

\usepackage{fancyhdr}
\setlength{\headheight}{15.71667pt} % Increased as recommended by fancyhdr
\addtolength{\topmargin}{-1.21667pt} % Optional: adjust topmargin as suggested
\usepackage{lastpage}

% For the chapters not to modify the page style
\usepackage{titlesec}
\titleformat{\chapter}{\bfseries}{\huge\arabic{chapter}.}{10pt}{\huge}
\titleclass{\chapter}{straight}

\usepackage{siunitx, multirow, booktabs}
\usepackage{blindtext,alltt}
\usepackage{csquotes} % Recommended for biblatex with babel/polyglossia
\usepackage[backend=biber, style=numeric, sorting=none, locallabelwidth]{biblatex}
\addbibresource{ThesisReferences.bib}

\usepackage{listings}

\usepackage{wrapfig}
\usepackage{array}
\usepackage{url}
\usepackage[nottoc]{tocbibind} %Add references to index.
\usepackage{amsmath, amssymb}
\usepackage{upgreek, dsfont, mathrsfs}
\usepackage{stmaryrd}

\usepackage{multicol}

% Estilo compacto para secciones en columnas
\titlespacing{\section}{0pt}{1ex plus 0.6ex minus 0.2ex}{0.8ex}
\titlespacing{\subsection}{0pt}{0.8ex plus 0.4ex minus 0.2ex}{0.5ex}
\titlespacing{\subsubsection}{0pt}{0.6ex plus 0.3ex minus 0.1ex}{0.4ex}

% Configuración de columnas tipo revista
\setlength{\columnsep}{25pt}           % separación entre columnas (default ~35pt)
\setlength{\columnseprule}{0pt}      % línea entre columnas (0pt = sin línea)

\usepackage{parskip} % Stop auto-indenting (to mimic markdown behaviour)
\usepackage[breakable]{tcolorbox}
% Compactar el interlineado en columnas (opcional)
\newenvironment{tightmulticols}{%
  \begin{multicols}{2}
  \setlength{\parskip}{0pt}
  \setlength{\parindent}{0em}
  \linespread{1}\selectfont
}{%
  \end{multicols}
}

% Basic figure setup, for now with no caption control since it's done
% automatically by Pandoc (which extracts ![](path) syntax from Markdown).
\usepackage{graphicx}
\usepackage{caption}

%Footnoe symbol instead of numbers
%1   asterisk        *   2   dagger      †   3   double dagger       ‡
%4   section symbol  §   5   paragraph   ¶   6   parallel lines      ‖
%7   two asterisks   **  8   two daggers ††  9   two double daggers  ‡‡
\usepackage[symbol]{footmisc}
\renewcommand{\thefootnote}{\fnsymbol{footnote}}

\usepackage{float}
\floatplacement{figure}{H} % forces figures to be placed at the correct location

\usepackage{xcolor} % Allow colors to be defined
\usepackage{enumerate} % Needed for markdown enumerations to work
\usepackage{textcomp} % defines textquotesingle
% Hack from http://tex.stackexchange.com/a/47451/13684:
\AtBeginDocument{%
	\def\PYZsq{\textquotesingle}% Upright quotes in Pygmentized code
}
\usepackage{upquote} % Upright quotes for verbatim code
\usepackage{eurosym} % defines \euro

\usepackage{iftex}
\ifPDFTeX
\IfFileExists{alphabeta.sty}{
	\usepackage{alphabeta}
}{
	\usepackage[mathletters]{ucs}
	\usepackage[utf8x]{inputenc}
}
\else
\usepackage{fontspec}
\usepackage{unicode-math}
\fi


% The hyperref package gives us a pdf with properly built
% internal navigation ('pdf bookmarks' for the table of contents,
% internal cross-reference links, web links for URLs, etc.)
\usepackage{hyperref}
% The default LaTeX title has an obnoxious amount of whitespace. By default,
% titling removes some of it. It also provides customization options.
\usepackage{titling}
\usepackage{longtable} % longtable support required by pandoc >1.10
\usepackage{calc}      % table minipage width calculation for pandoc >= 2.11.1
\usepackage[inline]{enumitem} % IRkernel/repr support (it uses the enumerate* environment)
\usepackage[normalem]{ulem} % ulem is needed to support strikethroughs (\sout)
% normalem makes italics be italics, not underlines
\usepackage{mathrsfs}

% More styles for bullets
\usepackage{pifont}

% Colors for the hyperref package
\definecolor{urlcolor}{rgb}{0,.145,.698}
\definecolor{linkcolor}{rgb}{.71,0.21,0.01}
\definecolor{citecolor}{rgb}{.12,.54,.11}

% ANSI colors
\definecolor{ansi-black}{HTML}{3E424D}
\definecolor{ansi-black-intense}{HTML}{282C36}
\definecolor{ansi-red}{HTML}{E75C58}
\definecolor{ansi-red-intense}{HTML}{B22B31}
\definecolor{ansi-green}{HTML}{00A250}
\definecolor{ansi-green-intense}{HTML}{007427}
\definecolor{ansi-yellow}{HTML}{DDB62B}
\definecolor{ansi-yellow-intense}{HTML}{B27D12}
\definecolor{ansi-blue}{HTML}{208FFB}
\definecolor{ansi-blue-intense}{HTML}{0065CA}
\definecolor{ansi-magenta}{HTML}{D160C4}
\definecolor{ansi-magenta-intense}{HTML}{A03196}
\definecolor{ansi-cyan}{HTML}{60C6C8}
\definecolor{ansi-cyan-intense}{HTML}{258F8F}
\definecolor{ansi-white}{HTML}{C5C1B4}
\definecolor{ansi-white-intense}{HTML}{A1A6B2}
\definecolor{ansi-default-inverse-fg}{HTML}{FFFFFF}
\definecolor{ansi-default-inverse-bg}{HTML}{000000}

% common color for the border for error outputs.
\definecolor{outerrorbackground}{HTML}{FFDFDF}

% Prevent overflowing lines due to hard-to-break entities
\sloppy 
% Setup hyperref package
\hypersetup{
	breaklinks=true,  % so long urls are correctly broken across lines
	colorlinks=true,
	urlcolor=urlcolor,
	linkcolor=linkcolor,
	citecolor=citecolor,
}

\definecolor{codegreen}{rgb}{0,0.6,0}
\definecolor{codegray}{rgb}{0.5,0.5,0.5}
\definecolor{codepurple}{rgb}{0.58,0,0.82}
\definecolor{backcolour}{rgb}{0.95,0.95,0.92}

\lstdefinestyle{mystyle}{
	backgroundcolor=\color{backcolour},   
	commentstyle=\color{codegreen},
	keywordstyle=\color{magenta},
	numberstyle=\tiny\color{codegray},
	stringstyle=\color{codepurple},
	basicstyle=\ttfamily\footnotesize,
	breakatwhitespace=false,         
	breaklines=true,                 
	captionpos=b,                    
	keepspaces=true,                 
	numbers=left,                    
	numbersep=5pt,                  
	showspaces=false,                
	showstringspaces=false,
	showtabs=false,                  
	tabsize=2
}

\lstset{style=mystyle}

\pagestyle{fancy}
\fancyhf{}
\rhead{}
\lhead{APRENDIZAJE PROFUNDO $|$ \textcolor{orange}{DETECCIÓN DE LA ABEJA REINA EN LA COLMENA}}
\lfoot{Universidad Autónoma de Chihuahua}
\rfoot{Page \thepage \hspace{1pt} of \pageref{LastPage}}

\renewcommand{\headrulewidth}{0.5pt} %ancho de la recta del encabezado superior

\begin{document}
	
	\begin{titlepage}
		\begin{center}
			\begin{tabular}{c}
				\includegraphics[scale=0.2]{BN_uach.png}\\[3.5ex]
				\textbf{\LARGE Universidad Autónoma de Chihuahua}\\[3.5ex]
				\textbf{\Large Facultad de Ingeniería}\\[3.5ex]
				\hline\\[3ex]
				\begin{minipage}{17cm}
					\centering
					\begin{doublespace}
						\textbf{\LARGE Detección de la Abeja Reina en una Colmena por Medio del Aprendizaje Profundo}\\[3.5ex]
					\end{doublespace}
				\end{minipage}\\[3.5ex]
				\hline
			\end{tabular}\vfill
			{\large Análisis de Frecuencias Emitidas por el Panal}\\\vfill
			{\large \textbf{Student:} Leonardo Rafael León Mora.}\\\vfill
			{\large \textbf{Thesis director:} Dr. Daniel Espinobarro Velázquez.}\\\vfill
			{\large \textbf{Advisors:}}\\[3.5ex]
			\begin{itemize}
				\item[\ding{226}] {\large M.C. Carlos Hugo Larrinúa Pacheco.}\\[3.5ex]
				\item[\ding{226}] {\large M.I. Joseph Isaac Ramírez Hernández.}\\[3.5ex]
			\end{itemize}
			\vfill
			{\large \textbf{Chihuahua, Chih.,} \today.}\\[3.5ex]
		\end{center}
	\end{titlepage}

%-------------------------------------------------------------------------------------
% 1. ÍNDICE
%-------------------------------------------------------------------------------------
\tableofcontents
\thispagestyle{empty} % No header/footer on the table of contents page


%-------------------------------------------------------------------------------------
% 2. RESUMEN/ABSTRACT
%-------------------------------------------------------------------------------------
\pagebreak
\chapter*{RESUMEN}
\addcontentsline{toc}{chapter}{RESUMEN}

%-------------------------------------------------------------------------------------
% 3. INTRODUCCIÓN
%-------------------------------------------------------------------------------------
\pagebreak
\chapter{INTRODUCCIÓN}
\vspace{-3em}

\begin{tightmulticols}
Los polinizadores, como las abejas melíferas, abejorros, abejas solitarias, mariposas, colibríes y murciélagos, desempeñan un papel fundamental en los ecosistemas, ya que contribuyen a la fertilización de plantas y cultivos. Las abejas melíferas, en particular, han sido ampliamente estudiadas por su destacada participación en la polinización de cultivos que forman parte de nuestra alimentación.

\par Durante las temporadas de mayor producción de miel, como primavera y verano, las abejas melíferas pueden recorrer distancias de hasta 6 km \cite{Beekman2000}, viajando 3 km de ida y vuelta entre la colmena y fuentes de alimento como flores, para recolectar polen, néctar o agua. Pero, ¿por qué surge este comportamiento? ¿Cuál es la motivación para las abejas en cuanto a regresar a la colmena y surtir la misma?

\par Según Ruvinga \cite{ruvinga2021use}, ``\textit{La colonia está formada por la reina, miles de abejas obreras estériles, unos pocos cientos de zánganos, huevos, larvas y pupas}''. Las abejas trabajan a diario, realizando tareas que se dividen no solo entre machos y hembras, sino también entre abejas jóvenes y adultas \cite{johnson2010division}. Un panal puede albergar entre 30,000 y 50,000 abejas, aunque este número puede reducirse hasta 5,000 durante el invierno. Este comportamiento de abastecimiento constante está estrechamente vinculado a la estructura social de la colmena y, en particular, al papel de la abeja reina, que es la única capaz de procrear \cite{ruvinga2021use}.

%-------------------------------------------------------------------------------------
% 3.1. Planteamiento del problema
%-------------------------------------------------------------------------------------

\section{Planteamiento del Problema}

\par La reina puede poner en promedio hasta 2,000 huevos al día. Las obreras se encargan de construir, defender y abastecer la colmena con recursos esenciales como agua, polen y néctar \cite{winston1987temperate}. Los zánganos cumplen funciones reproductivas y también se cree que ayudan a regular la temperatura del panal, mientras que las abejas más jóvenes se limitan a tareas dentro de la colmena \cite{johnson2010division}. La pérdida de la abeja reina puede comprometer seriamente la supervivencia de la colonia.

\par Las colmenas están expuestas a múltiples amenazas: insectos invasores, pesticidas, condiciones climáticas y hasta errores humanos. En sistemas apícolas tradicionales donde se utilizan cajas de madera, cada inspección física implica abrir la colmena y romper el sello de propóleo que las abejas han construido para mantener un ambiente limpio y controlado. Esta intervención frecuente no solo representa una pérdida energética para la colonia (que debe reconstruir el sello), sino que además incrementa el riesgo de aplastar accidentalmente a la reina durante la manipulación interna del panal. Esta situación puede conducir al colapso de toda la colonia.

%-------------------------------------------------------------------------------------
% 3.2. Justificación
%-------------------------------------------------------------------------------------

\section{Justificación}

\par Contar con herramientas que permitan identificar la presencia o ausencia de la abeja reina de manera no invasiva resulta fundamental para garantizar la continuidad y productividad de la colmena. Esta investigación propone un enfoque basado en el análisis acústico del panal, aprovechando técnicas de aprendizaje automático para detectar el estado de la reina. Dado que estudios recientes han demostrado que el sonido dentro de la colmena varía según la presencia o ausencia de la reina \cite{ruvinga2021use, maarefremote, hunter2013signal, ruvinga2023identifying}, se busca validar un sistema que, a partir de grabaciones de audio, identifique estas variaciones de forma automática y eficiente.

%-------------------------------------------------------------------------------------
% 3.3. Objetivos
%-------------------------------------------------------------------------------------

\section{Objetivos}

\subsection{Objetivo General}

\par Desarrollar un programa de monitoreo de frecuencias con el que se pueda determinar si una abeja reina se encuentra en la colmena, mediante el análisis de grabaciones de audio y técnicas de aprendizaje automático, a partir del cual se pueda proponer un sistema no invasivo de seguimiento para determinar el estado de la colmena.

\subsection{Objetivos Específicos}

\begin{itemize}
    \item[\ding{43}] Realizar una revisión bibliográfica sobre la estructura de las colonias de abejas melíferas y los efectos de la ausencia de la reina.
    \item[\ding{43}] Procesar y organizar un conjunto de datos de audio de colmenas etiquetado con el estado de la reina.
    \item[\ding{43}] Extraer características acústicas relevantes, como los coeficientes cepstrales en la escala Mel (MFCC), a partir de las grabaciones.
    \item[\ding{43}] Entrenar y evaluar modelos de aprendizaje automático para clasificar el estado de la reina basado en características extraídas.
    \item[\ding{43}] Validar la efectividad del sistema en términos de precisión, sensibilidad y especificidad.
\end{itemize}

\section{Hipótesis de Trabajo}

\par Dado que el comportamiento sonoro de una colmena varía según el estado de su reina, se espera que un sistema basado en el análisis de características acústicas sea capaz de detectar, con un grado aceptable de precisión, si la reina está presente, ausente, rechazada o recién aceptada en la colonia.

\end{tightmulticols}

%-------------------------------------------------------------------------------------
% 4. MARCO TEÓRICO
%-------------------------------------------------------------------------------------

\pagebreak
\chapter{MARCO TEÓRICO}
\vspace{-3em}

\begin{tightmulticols}

%-------------------------------------------------------------------------------------
% 4.1. Definición de conceptos clave
%-------------------------------------------------------------------------------------

\section{Definición de Conceptos Clave}

\begin{itemize}
    \item[\ding{43}] \textbf{Colmena}: estructura donde viven las abejas melíferas, compuesta por una reina, obreras y zánganos.
	\item[\ding{43}] \textbf{Reina}: abeja hembra fértil encargada de la reproducción en la colmena.
	\item[\ding{43}] \textbf{Razón de muestreo (Sample Rate)}: número de muestras de audio tomadas por segundo (cuyos valores para el monitoreo de abejas comunmente rondan entre los 16000 a 22050 Hz).
	\item[\ding{43}] \textbf{MFCC (Mel-Frequency Cepstral Coefficients)}: características acústicas utilizadas para representar el contenido espectral de una señal de audio, especialmente útiles en el reconocimiento de patrones sonoros.
\end{itemize}

%-------------------------------------------------------------------------------------
% Revisión de literatura
%-------------------------------------------------------------------------------------

%-------------------------------------------------------------------------------------
% Fundamentos teóricos
%-------------------------------------------------------------------------------------


\end{tightmulticols}

%-------------------------------------------------------------------------------------
% 5. METODOLOGÍA
%-------------------------------------------------------------------------------------

\pagebreak
\chapter{METODOLOGÍA}
\vspace{-3em}

% Crear una sección al final con los programas utilizados y hacer referencia a ella en el texto de esta sección.

\begin{tightmulticols}
Inicialmente se determinó el formato de los archivos de audio y la razón de muestreo [\ref{lst:waveform}]. En este estudio, se utiliza una razón de muestreo de 16 kHz con un formato de mono-audio (1 canal).

%-------------------------------------------------------------------------------------
% 5.1. Diseño de investigación
%-------------------------------------------------------------------------------------
\section{Diseño de Investigación}

\par Este estudio se enmarca dentro de un diseño de investigación aplicada, con un enfoque cuantitativo. Se busca desarrollar un sistema que permita la detección automática del estado de la abeja reina en una colmena a partir del análisis de grabaciones de audio. Para ello, se emplearán técnicas de aprendizaje automático y procesamiento de señales acústicas.

%-------------------------------------------------------------------------------------
% 5.2. Instrumentos y técnicas
%-------------------------------------------------------------------------------------
\section{Instrumentos y Técnicas}

\par El proceso de recolección de datos implicó la grabación de audio en condiciones controladas, asegurando que las grabaciones fueran representativas del comportamiento sonoro de las colmenas. Los datos fueron adquiridos utilizando un dispositivo IoT que incluía un módulo Wi-Fi ESP32, un micrófono digital INMP441 y un sensor de temperatura y humedad BME280. Estos sensores fueron colocados dentro de la cubierta telescópica de la colmena —por encima de los cuadros pero por debajo de la cubierta exterior— y se encerraron con una malla metálica para evitar la interferencia de las abejas con los componentes electrónicos, especialmente el micrófono.

%-------------------------------------------------------------------------------------
% 5.3. Población y muestra (si aplica)
%-------------------------------------------------------------------------------------

\section{Base de Datos}

El conjunto de datos utilizado en esta investigación fue grabado y publicado por Anna Yang\cite{AnnaYang-Dataset}. Contiene datos de audio recolectados de colmenas de abejas melíferas europeas ubicadas en California. Las grabaciones fueron segmentadas en clips de 60 segundos, resultando en un total de 7,100 muestras recolectadas durante un período de cinco semanas, desde el 8 de junio de 2022 (14:52:28) hasta el 15 de julio de 2022 (15:28:21). Cada día contiene aproximadamente entre 23 y 24 muestras.

El conjunto de datos contiene múltiples características de metadatos que proporcionan información sobre el entorno de la colmena y la condición de la colonia. De particular relevancia para este estudio se encuentran las siguientes etiquetas:

\begin{itemize}
	\item[\ding{43}] \texttt{queen acceptance}: indica si la reina es aceptada por la colonia. Valores: 0 (sin reina), 1 (presente pero no aceptada), 2 (aceptada).
	\item[\ding{43}] \texttt{queen presence}: indica si la reina está físicamente presente en la colmena.
	\item[\ding{43}] \texttt{queen status}: una etiqueta combinada derivada de las dos anteriores, con los siguientes valores: 0 (original/con reina funcional), 1 (no presente), 2 (presente y rechazada), 3 (presente y recién aceptada). Esta es la etiqueta principal utilizada para la clasificación en este trabajo.
	\item[\ding{43}] \texttt{time}: la hora del día, escalada entre 0 y 1 para representar un ciclo de 24 horas.
\end{itemize}

El proceso de etiquetado se basó en inspecciones manuales diarias de las colmenas, donde se observó el comportamiento de las abejas hacia la reina (por ejemplo, formación de enjambre o comportamiento tranquilo). Para aquellos casos en los que el estado de la reina cambió entre inspecciones (es decir, había ambigüedad sobre el momento exacto de aceptación), las muestras fueron descartadas para mantener la integridad de las etiquetas.

\end{tightmulticols}

%-------------------------------------------------------------------------------------
% 6. RESULTADOS
%-------------------------------------------------------------------------------------

\pagebreak
\chapter{RESULTADOS}
\vspace{-3em}

%-------------------------------------------------------------------------------------
% 6.1. Análisis de datos
%-------------------------------------------------------------------------------------

%-------------------------------------------------------------------------------------
% 6.2. Presentación de resultados
%-------------------------------------------------------------------------------------

%-------------------------------------------------------------------------------------
% 7. DISCUSIÓN
%-------------------------------------------------------------------------------------

\pagebreak
\chapter{DISCUSIÓN}
\vspace{-3em}

%-------------------------------------------------------------------------------------
% 7.1. Interpretación de resultados
%-------------------------------------------------------------------------------------

%-------------------------------------------------------------------------------------
% 7.2. Comparación con la literatura
%-------------------------------------------------------------------------------------

%-------------------------------------------------------------------------------------
% 8. CONCLUSIONES Y RECOMENDACIONES
%-------------------------------------------------------------------------------------
\pagebreak
\chapter{CONCLUSIONES Y RECOMENDACIONES}

%-------------------------------------------------------------------------------------
% 8.1. Conclusiones
%-------------------------------------------------------------------------------------

\section{Conclusiones}



%-------------------------------------------------------------------------------------
% 8.2. Recomendaciones
%-------------------------------------------------------------------------------------

%-------------------------------------------------------------------------------------
% 9. REFERENCIAS BIBLIOGRÁFICAS
%-------------------------------------------------------------------------------------

\pagebreak
\addcontentsline{toc}{chapter}{BIBLIOGRAFÍA}
\printbibliography
\thispagestyle{empty}

%-------------------------------------------------------------------------------------
% 10. ANEXOS (opcional)
%-------------------------------------------------------------------------------------

\pagebreak
\addcontentsline{toc}{chapter}{APÉNDICES}
\begin{center}
\large\textbf{APÉNDICE A}
\end{center}

\begin{lstlisting}[language=Python, caption={Visualización de la forma de onda de un archivo de audio.}, label={lst:waveform}]
import numpy as np
import matplotlib.pyplot as plt
from scipy.io import wavfile

# cargar archivo de audio
sr, audio = wavfile.read('2022-06-08--14-52-28_1__segment3.wav')

# normalizar si es int16
if audio.dtype == np.int16:
    audio = audio.astype(np.float32) / np.iinfo(np.int16).max

# crear eje de tiempo
duration_sec = len(audio) / sr
time_axis = np.linspace(0, duration_sec, num=len(audio))

# graficar la onda
plt.figure(figsize=(12, 4))
plt.plot(time_axis, audio)
plt.title(f'Onda (Sample Rate: {sr} Hz)')
plt.xlabel('Tiempo (segundos)')
plt.ylabel('Amplitud')
plt.grid(True)
plt.tight_layout()
plt.show()

\end{lstlisting}

\end{document}