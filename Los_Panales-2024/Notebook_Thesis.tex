% !TeX spellcheck = en_US
\documentclass[12pt]{report}
\usepackage[a4paper, total={7in, 9in}]{geometry}
\usepackage{setspace} %Line spacing
\onehalfspacing

\usepackage{pdfpages}

\usepackage{fancyhdr}
\usepackage{lastpage}

% For the chapters not to modify the page style
\usepackage{titlesec}
\titleformat{\chapter}{\bfseries}{\huge\arabic{chapter}.}{10pt}{\huge}
\titleclass{\chapter}{straight}

%REFERENCES
\usepackage{siunitx, multirow, booktabs}
\usepackage{blindtext,alltt}
\usepackage[backend=bibtex, style=numeric, sorting=none, locallabelwidth]{biblatex}
\addbibresource{ThesisReferences.bib}

\usepackage{listings}

\usepackage{wrapfig}
\usepackage{array}
\usepackage{multicol}
\usepackage{url}
\usepackage[nottoc]{tocbibind} %Add references to index.
\usepackage{amsmath, amssymb}
\usepackage{upgreek, dsfont, mathrsfs}
\usepackage{stmaryrd}


\usepackage[breakable]{tcolorbox}
\usepackage{parskip} % Stop auto-indenting (to mimic markdown behaviour)


% Basic figure setup, for now with no caption control since it's done
% automatically by Pandoc (which extracts ![](path) syntax from Markdown).
\usepackage{graphicx}
\usepackage{caption}

%Footnoe symbol instead of numbers
%1   asterisk        *   2   dagger      †   3   double dagger       ‡
%4   section symbol  §   5   paragraph   ¶   6   parallel lines      ‖
%7   two asterisks   **  8   two daggers ††  9   two double daggers  ‡‡
\usepackage[symbol]{footmisc}
\renewcommand{\thefootnote}{\fnsymbol{footnote}}

\usepackage{float}
\floatplacement{figure}{H} % forces figures to be placed at the correct location

\usepackage{xcolor} % Allow colors to be defined
\usepackage{enumerate} % Needed for markdown enumerations to work
\usepackage{geometry} % Used to adjust the document margins
\usepackage{amsmath} % Equations
\usepackage{amssymb} % Equations
\usepackage{textcomp} % defines textquotesingle
% Hack from http://tex.stackexchange.com/a/47451/13684:
\AtBeginDocument{%
	\def\PYZsq{\textquotesingle}% Upright quotes in Pygmentized code
}
\usepackage{upquote} % Upright quotes for verbatim code
\usepackage{eurosym} % defines \euro

\usepackage{iftex}
\ifPDFTeX
\IfFileExists{alphabeta.sty}{
	\usepackage{alphabeta}
}{
	\usepackage[mathletters]{ucs}
	\usepackage[utf8x]{inputenc}
}
\else
\usepackage{fontspec}
\usepackage{unicode-math}
\fi


% The hyperref package gives us a pdf with properly built
% internal navigation ('pdf bookmarks' for the table of contents,
% internal cross-reference links, web links for URLs, etc.)
\usepackage{hyperref}
% The default LaTeX title has an obnoxious amount of whitespace. By default,
% titling removes some of it. It also provides customization options.
\usepackage{titling}
\usepackage{longtable} % longtable support required by pandoc >1.10
\usepackage{booktabs}  % table support for pandoc > 1.12.2
\usepackage{array}     % table support for pandoc >= 2.11.3
\usepackage{calc}      % table minipage width calculation for pandoc >= 2.11.1
\usepackage[inline]{enumitem} % IRkernel/repr support (it uses the enumerate* environment)
\usepackage[normalem]{ulem} % ulem is needed to support strikethroughs (\sout)
% normalem makes italics be italics, not underlines
\usepackage{mathrsfs}



% Colors for the hyperref package
\definecolor{urlcolor}{rgb}{0,.145,.698}
\definecolor{linkcolor}{rgb}{.71,0.21,0.01}
\definecolor{citecolor}{rgb}{.12,.54,.11}

% ANSI colors
\definecolor{ansi-black}{HTML}{3E424D}
\definecolor{ansi-black-intense}{HTML}{282C36}
\definecolor{ansi-red}{HTML}{E75C58}
\definecolor{ansi-red-intense}{HTML}{B22B31}
\definecolor{ansi-green}{HTML}{00A250}
\definecolor{ansi-green-intense}{HTML}{007427}
\definecolor{ansi-yellow}{HTML}{DDB62B}
\definecolor{ansi-yellow-intense}{HTML}{B27D12}
\definecolor{ansi-blue}{HTML}{208FFB}
\definecolor{ansi-blue-intense}{HTML}{0065CA}
\definecolor{ansi-magenta}{HTML}{D160C4}
\definecolor{ansi-magenta-intense}{HTML}{A03196}
\definecolor{ansi-cyan}{HTML}{60C6C8}
\definecolor{ansi-cyan-intense}{HTML}{258F8F}
\definecolor{ansi-white}{HTML}{C5C1B4}
\definecolor{ansi-white-intense}{HTML}{A1A6B2}
\definecolor{ansi-default-inverse-fg}{HTML}{FFFFFF}
\definecolor{ansi-default-inverse-bg}{HTML}{000000}

% common color for the border for error outputs.
\definecolor{outerrorbackground}{HTML}{FFDFDF}

% Prevent overflowing lines due to hard-to-break entities
\sloppy 
% Setup hyperref package
\hypersetup{
	breaklinks=true,  % so long urls are correctly broken across lines
	colorlinks=true,
	urlcolor=urlcolor,
	linkcolor=linkcolor,
	citecolor=citecolor,
}

\definecolor{codegreen}{rgb}{0,0.6,0}
\definecolor{codegray}{rgb}{0.5,0.5,0.5}
\definecolor{codepurple}{rgb}{0.58,0,0.82}
\definecolor{backcolour}{rgb}{0.95,0.95,0.92}

\lstdefinestyle{mystyle}{
	backgroundcolor=\color{backcolour},   
	commentstyle=\color{codegreen},
	keywordstyle=\color{magenta},
	numberstyle=\tiny\color{codegray},
	stringstyle=\color{codepurple},
	basicstyle=\ttfamily\footnotesize,
	breakatwhitespace=false,         
	breaklines=true,                 
	captionpos=b,                    
	keepspaces=true,                 
	numbers=left,                    
	numbersep=5pt,                  
	showspaces=false,                
	showstringspaces=false,
	showtabs=false,                  
	tabsize=2
}

\lstset{style=mystyle}

\pagestyle{fancy}
\fancyhf{}
\rhead{}
\lhead{NOTES $|$ \textcolor{red}{THESIS RELATED TOPICS}}
\lfoot{Universidad Autónoma de Chihuahua}
\rfoot{Page \thepage \hspace{1pt} of \pageref{LastPage}}

\renewcommand{\headrulewidth}{0.5pt} %ancho de la recta del encabezado superior


\begin{document}
	
	%\thispagestyle{empty}
	\begin{titlepage}
		\begin{center}
			\begin{tabular}{c}
				\includegraphics[scale=0.2]{BN_uach.png}\\[3.5ex]
				\textbf{\LARGE Universidad Autónoma de Chihuahua}\\[3.5ex]
				\textbf{\Large Facultad de Ingeniería}\\[3.5ex]
				\hline\\[3ex]
				\begin{minipage}{17cm}
					\centering
					\begin{doublespace}
						\textbf{\LARGE NOTEBOOK: Development of Topics Related to the Project}
					\end{doublespace}
				\end{minipage}\\[3.5ex]
				\hline
			\end{tabular}\vfill
			{\large Notes Before Start Working on the Thesis}\\\vfill
			{\large \textbf{Student:} Leonardo Rafael León Mora.}\\\vfill
			{\large \textbf{Thesis director:} Dr. Daniel Espinobarro Velázquez.}\\\vfill
			{\large \textbf{Advisors:}}\\[3.5ex]
			\begin{itemize}
				\item {\large M.C. Carlos Hugo Larrinúa Pacheco.}\\[3.5ex]
				\item {\large M.I. Joseph Isaac Ramírez Hernández.}\\[3.5ex]
			\end{itemize}
			\vfill
			{\large \textbf{Chihuahua, Chih.,} April 14, 2024.}\\[3.5ex]
		\end{center}
	\end{titlepage}

	\part*{BASIC RECAP}
	
	\chapter*{DEFINITIONS}


        %-------------------------------------------------------------------------------------
	% 1. Differential Equations - Notes
	%-------------------------------------------------------------------------------------
	
	\chapter{DIFFERENTIAL EQUATIONS}

        \par \textbf{DEFINITION 1.1.1 - Differential Equation}

        \par An equation containing the derivatives of one or more unknown functions (or dependent variables), with respect to one or more independent variables, is said to be a \textbf{differential equation (DE)}.

        \section{Notation}
        \begin{itemize}
            \item \textbf{Leibinz notation}:
            \begin{align*}
                \frac{dy}{dx}, \; \frac{d^{2}y}{dx^{2}}, \; \frac{d^{3}y}{dx^{3}}, \; ...
            \end{align*}
            \item \textbf{Prime notation}:
            \begin{align*}
                y', \; y'', \; y''', \; y^{(4)}, \; ...
            \end{align*}
        \end{itemize}
            In general, the $n$th derivative of y is written as $\frac{d^{n}y}{dx^{n}}$ or $y^{(n)}$.
        \begin{itemize}
            \item \textbf{Newton's dot notation}: used in physical science and engineering, to denote derivatives with respect to time $t$,
            \begin{align*}
                \frac{d^{2}s}{dt^{2}} = -32 \; \rightarrow \; \text{becomes} \; \rightarrow \; \ddot{s} = -32.
            \end{align*}
            \item \textbf{Subscript notation}: often used in partial derivatives indicating the independent variables.
            \begin{align*}
                \frac{\partial^{2} u}{\partial x^{2}} = \frac{\partial^{2} u}{\partial t^{2}} - 2 \frac{\partial u}{\partial t}\; \rightarrow \; \text{becomes} \; \rightarrow \; u_{xx} = u_{tt} - 2 u_{t}.
            \end{align*}
        \end{itemize}

        \section{Classification By Type}

        \par If a differential equation contains only ordinary derivatives of one or more unknown functions with respect to a single independent variable, it is said to be an \textbf{ordinary differential equation (ODE)}. An equation involving partial derivatives of one or more unknown functions of two or more independent variables is called a \textbf{partial differential equation (PDE)}.

        \section{Classification By Order}

        \par The order of a differential equation (either ODE or PDE) is the order of the highest derivative in the equation. For example,
        \begin{align*}
            \frac{d^{2}y}{dx^{2}} + 5 \left( \frac{dy}{dx} \right)^3 - 4y = e^{x} \; ,
        \end{align*}
        \par is a second-order differential equation.
    
	%-------------------------------------------------------------------------------------
	% 2. Maths of Physics - Notes
	%-------------------------------------------------------------------------------------
	
	\part{MATHEMATICS OF PHYSICS}
	
	\chapter{INFINITE SERIES, POWER SERIES}
	
	\section{The Geometric Series\cite{MaryBoas-MathofPhysics-2006}}
	
	\par \textbf{Geometric Progression}, is a sequence of numbers where each term after the first is found by multiplying the previous term by a fixed, non-zero number called the common ratio. The \textbf{general form} of a geometric progression is:
	\begin{equation}\label{eq:GP_general-form-1}
		a, \, ar, \, ar^{2}, \, ar^{3}, \, ...
	\end{equation}
	where $a$ is the \textit{first term} and $r$ is the \textit{common ratio}. Geometric Progressions have some interesting properties:
	\begin{itemize}
		\item If $r > 1$, the terms will increase.
		\item If $0 < r < 1$, the terms will decrease.
		\item If $r = 1$ the sequence becomes constant (where each term is the same).
		\item If $r < 0$, the terms will alternate between positive and negative values.
	\end{itemize} 
	i.e., Suppose a bouncing ball rises each time to $\frac{2}{3}$ of the height of the previous bounce. Then	
	\begin{equation}\label{BounceBall-ie1}
		1, \, \frac{2}{3}, \, \frac{4}{9}, \, \frac{8}{27}, \, \frac{16}{81}, \, ... \, ,
	\end{equation}
	
	%-------------------------------------------------------------------------------------
	% 3. Numerical Methods - Notes
	%-------------------------------------------------------------------------------------
	
	\part{NUMERICAL METHODS}
	
	\chapter{INTRODUCTION}

        \par The term \textbf{Numerical Methods} refers to those techniques used to approximate the solution to a mathematical problem, these methods are used for mathematical processes such as integrals, differential equations, nonlinear equations in which the solution is close to the exact one and the error of that result can also be quantified by an approximation.

        \section{Steps for solving an Engineering Problem}

        \begin{itemize}
        \item \textbf{Description}: present the problem exposing every single need related to it, write the background of it and the need for its solution.
        \item \textbf{Mathematical Model}: present an equation that best describes the situation when fitting the given data.
        \item \textbf{Solution of the Mathematical Model}: find the solution to the proposed equation.
        \item \textbf{Using the Solution}: propose a solution to the real problem based on the obtained result from the mathematical model.
        \end{itemize}
	
	\chapter{MATHEMATICAL PROCESSES}

        \section{Roots of Nonlinear Equations}

        \section{Simultaneous Linear Equations}

        \section{Curve Fitting by Interpolation}

        \section{Differentiation}

        \section{Curve Fitting by Regression}

        \section{Numerical Integration}

        \section{Ordinary Differential Equations}

        \section{Partial Differential Equations}

        \section{Optimization}

        \section{Fast Fourier Transform}
 
	%-------------------------------------------------------------------------------------
	% BIBLIOGRAFÍA.
	%-------------------------------------------------------------------------------------
	\pagebreak
	\addcontentsline{toc}{chapter}{BIBLIOGRAPHY}
	\printbibliography
	\thispagestyle{empty}
	
\end{document}